\documentclass[12pt]{article}
\setlength{\textwidth}{17cm}
\setlength{\textheight}{24cm}
\setlength{\topmargin}{-2cm}
\setlength{\footskip}{1cm}
\setlength{\evensidemargin}{0cm}
\setlength{\oddsidemargin}{0cm}
\setlength{\parindent}{0cm}

\usepackage{allrunes}
\usepackage{amsmath}
\usepackage[magyar]{babel}
\usepackage[T1]{fontenc}
\usepackage[utf8]{inputenc}
\usepackage{fixltx2e}
\usepackage{multirow}

\usepackage[hyphens]{url}
\usepackage[unicode,colorlinks=true,breaklinks]{hyperref}
%\usepackage[dvips]{hyperref}
%should display links, but it does not work with \H accent
%and formulas in section titles

\hypersetup{colorlinks,linkcolor=blue,urlcolor=magenta,citecolor=magenta}
%Breaks long url`s in text, while keeping it one link:

\usepackage{amsfonts}
\usepackage{amsthm}
\usepackage{amssymb}


\theoremstyle{plain}
\usepackage{graphicx}

%\usepackage{gensymb}
\usepackage{float}

% For bra-ket notation
\usepackage{braket}

%% New commands
\newcommand{\dd}{\textrm{d}}

%% Pauli matrices
\newcommand{\sigx}{\sigma_x}
\newcommand{\sigy}{\sigma_y}
\newcommand{\sigz}{\sigma_z}

\newcommand{\paulix}{
    \left( \begin{array}{cc}
        0 & 1 \\
        1 & 0
    \end{array}
    \right)
}

\newcommand{\pauliy}{
    \left( \begin{array}{cc}
        0 & -i \\
        i & 0
    \end{array}
    \right)
}

\newcommand{\pauliz}{
    \left( \begin{array}{cc}
        1 & 0 \\
        0 & -1
    \end{array}
    \right)
}


\begin{document}
\title{7. tétel}
\author{Horváth Benedek}

\maketitle


\newpage
\begin{abstract}
    Molekuláris dinamika, Verlet- és sebesség-Verlet-algoritmus, termodinamikai mennyiségek meghatározása és relaxáció.
\end{abstract}


\section{Bevezetés}

A molekuladinamika hatékony és széles körben alkalmazott szimulációs eljárás kristályos, amorf, folyékony és gáz halmazállapotú anyagok, sőt, makromolekulák fizikai és kémiai tulajdonságainak meghatározására, illetve időfüggő mikroszkopikus folyamatok, mint például diffúzió modellezésére. A módszer hatékonysága abban rejlik, hogy az atomok belső -- kvantummechanikával leírható -- szerkezetét elhanyagolva azokat klasszikus és pontszerű objektumoknak tekintjük, és egyenként követjük a trajektóriájukat a klasszikus newtoni mozgásegyenletük numerikus integrálásával. A módszer tehát teljesen determinisztikus -- ellentétben a sokrészecske-szimulációk egy másik típusával, a Monte Carlóval. Az atomok között párkölcsönhatásokat veszünk figyelembe, amit valamilyen potenciállal írunk le. Az egyszerű Lennard-Jones potenciáltól kezdve az igen bonyolult, molekulák rotációs és vibrációs módusait, illetve belső polarizációját is modellező erőtereken át a kovalens kötések létrejöttét és felbomlását is figyelembe vevő reaktív erőterekig igen sokféle potenciált használnak a szimulációkban. Közös tulajdonságuk, hogy elméleti -- gyakran kvantummechanikai, pl. sűrűségfunkcionál-elmélet, Hartree-Fock módszer -- háttérszámolásokkal vannak validálva, többparaméteres függvényillesztés révén. (Részletekre itt most nem térek ki, az érdeklődők számára szemléltető példaként ajánlom a szén, hidrogén és oxigén atomokat tartalmazó rendszerek összetett Reactive Force Field típusú potenciáljáról szóló publikációt \cite{KimberlyChenoweth2008}.) A párkölcsönhatások számolása és a részecskék egyenkénti mozgatása komolyan limitálja a szimulálható részecskék számát, a mai számítástechnika mellett jellemzően $10^3-10^8$ közé. A részecskék dinamikáját követve elegendően hosszú számolás után a rendszer kollektív termodinaikai tulajdonságai is számolhatók a statisztikus fizika törvényeit alkalmazva.


\section{A dinamika szimulációja}

A molekuladinamikai szimulációban az egy részecskére ható erő az összes többi részecskével való párkölcsönhatásból származó erő eredője:

\begin{equation}
	m_i \frac{\dd^2 \mathbf{r}_i}{\dd t^2} = \sum_{j \neq i =1}^{N} \mathbf{f}_{ij},
\end{equation}
ahol $N$ az összes részecske száma a szimulációban. Másképp fogalmazva, a teljes rendszert leíró potenciál a párpotenciálok összege:

\begin{equation}
	U(\mathbf{r}_1, \mathbf{r}_2 \dots \mathbf{r}_N) = \sum_{i<j} u(r_{ij}) = 
	\sum_{i=1}^{N-1} \sum_{j=i+1}^{N} u(r_{ij}),
\end{equation}
ahol $r_{ij} = r_{ji} = |\mathbf{r}_i - \mathbf{r}_j|$ két részecske távolsága, $u(r_{ij})$ a köztük lévő távolságfüggő párpotenciál, az összegzésnél pedig minden párkölcsönhatást csak egyszer számolunk. A rendszert tehát egy konzervatív potenciál írja le, így a teljes energia (a kinetikus és potenciális energia összege) elviekben állandó. Az egy részecskére ható erő a teljes potenciálból egyszerűen kifejezve:

\begin{equation}
	\mathbf{F}_i (\mathbf{r}_1, \mathbf{r}_2 \dots \mathbf{r}_N) = -\nabla_{\mathbf{r}_i} U(\mathbf{r}_1, \mathbf{r}_2 \dots \mathbf{r}_N).
\end{equation}

Ugyanilyen logikával az egy elemi párkölcsönhatásból származó erő a párpotenciálból, komponensenként számolva:

\begin{equation}
	(f_{ij})_x = -\frac{\partial u(r_{ij})}{\partial x_i} = -\frac{\partial u(r_{ij})}{\partial r_{ij}} \frac{\partial r_{ij}}{\partial x_i},
\end{equation}
ahol 

\begin{equation}
	r_{ij} = ( (x_i - x_j)^2 + (y_i - y_j)^2 + (z_i - z_j)^2 )^{\frac{1}{2}},
\end{equation}
azaz

\begin{equation}
	\frac{\partial r_{ij}}{\partial x_i} = \frac{x_i - x_j}{r_{ij}}.
\end{equation}
A $j$ részecske által az $i$ részecskére ható erő tehát:

\begin{equation}
	\mathbf{f}_{ij} = -\frac{\dd u(r_{ij})}{\dd r_{ij}} \left( \begin{array}{c}
	\frac{x_i - x_j}{r_{ij}} \\ \\
	\frac{y_i - y_j}{r_{ij}} \\ \\
	\frac{z_i - z_j}{r_{ij}}
	\end{array}
	\right) = -\frac{\dd u(r_{ij})}{\dd r_{ij}} \hat{\mathbf{e}}_{ij},
\end{equation}
ahol $\hat{\mathbf{e}}_{ij}$ a $j$ részecske pozíciójából az $i$ felé mutató egységvektor. Látszik, hogy a hatás-ellenhatás törvényének megfelelően a párkölcsönhatásban résztvevő két részecskére ható erő csak előjelben különbözik, amit a szimulációban érdemes kihasználni. Mivel a részecskékre erő-ellenerő párok hatnak -- nincs külső erő --, a részecskék összimpulzusa elviekben zérus, így a rendszer tömegközéppontja nem mozoghat a szimuláció során. Szimulációs kód fejlesztése során ennek ellenőrzésével meggyőződhetünk az algoritmus és a numerikus integrálás helyességéről.

A párpotenciálra egy konkrét, egyszerű példa a vonzó-taszító Lennard-Jones potenciál, ami másodrendű kölcsönhatások leírására jól alkalmazható (a gyakorlatban például nemesgázokra, ahol nincsenek kovalens kötések és bonyolult, sok szabadsági fokú molekulák):

\begin{equation}
u(r) = 4\varepsilon \left[ \left(\frac{\sigma}{r}\right)^{12} - \left(\frac{\sigma}{r}\right)^{6} \right],
\end{equation}
ahol $\varepsilon$ a kölcsönhatás erősségét, $\sigma$ a hatótávolságát határozza meg, az $r^{-6}$ tag a nagy távolságokon fellépő van der Waals vonzó kölcsönhatást, az $r^{-12}$ tag pedig az elektronok Coulomb-taszításából és a Pauli-féle kizárási elvből kiindulva gyakorlatilag kemény gömbként veszi figyelembe az atomokat (lásd: \ref{fig:lj}. ábra). Bonyolultabb rendszerek leírására a kölcsönható részecskék típusától függően más és más, különféle tagok összegéből adódó párpotenciálokat használnak.


\begin{figure}
	\centering
	\includegraphics[width=0.7\linewidth]{media/LJ}
	\caption{A Lennard-Jones potenciál távolságfüggése $\sigma$ és $\varepsilon$ egységekben, a taszító és vonzó tag összegéből képezve. Ábra: \cite{Landau2012}.}
	\label{fig:lj}
\end{figure}



\subsection{A potenciál levágása}

A számítások egyszerűsítése érdekében a párpotenciálhoz minden esetben definiálnak egy levágási sugarat:

\begin{equation}
	u(r) = \begin{cases}
	\begin{array}{lr}
		4 (r^{-12} - r^{-6}), & \rm{ha} ~ r<r_{cut},\\
	    0,					  & \rm{ha} ~ r>r_{cut},
	\end{array}
	\end{cases}
\end{equation}

ahol szemléltető példaként továbbra is a Lennard-Jones potenciált használtuk.
Az egy részecskére ható erőhöz tehát csak a levágási sugáron belül lévő részecskék adnak járulékot. A potenciál így nem folytonos, az abból számolt erő (a potenciál deriváltja) pedig szinguláris a levágási sugár értékénél ($r= r_{cut}$). Az elviekben konzervatív potenciál így már nem az, vagyis az energiamegmaradás nem teljesül egzaktul. A gyakorlatban viszont a levágási sugarat kellőképpen nagynak választjuk ahhoz, hogy a környezetében az erő értéke elhanyagolható legyen az egyéb közelítési és numerikus hibákhoz képest, így az energiamegmaradás ne sérüljön lényegesen \cite{Landau2012}.


\subsection{Felületi jelenségek}

Molekuladinamikai szimulációk segítségével gyakran makroszkopikus rendszerek viselkedésére igyekszünk következtetni $10^{23}$ részecskeszámmal. Mivel ennek a részecskeszámnak csak a töredékét szimuláljuk egy véges térfogatban, lényeges a felületi jelenségek megfelelő kezelése\footnote{Minél kevesebb részecskét szimulálunk, annál nagyobb problémát jelenthetnek a mesterséges felületi jelenségek. Vegyünk például 1000 gázrészecskét egy $10\times 10 \times 10$ egység méretű kockában, azaz egy részecske pont egy térfogategységet töltsön be átlagosan. Így $10^3-8^3 = 488$ részecske a felület szomszédságában helyezkedik el, ez 49\%-a az összes részecskének. $10^6$ számú részecskénél ugyanilyen megfontolás mentén már csak a részecskék 6\%-a található a felület szomszédságában.}. Leggyakrabban periodikus határfeltételt alkalmazunk, az alábbiakban kizárólag ezt mutatom be. Ettől eltérő eset például a kapilláris csövekben modellezett dinamika, ahol kemény falat kell feltételezni, illetve biopolimerek szimulációjakor szükség lehet stochasztikus határfeltételre vagy határfeltétel nélküli (effektíve végtelen) szimulációs térfogatra \cite{lecture}. 

A periodikus határfeltétel esetében gyakorlatilag azt feltételezzük, hogy a szimulációban definiált térfogat/síkidom végtelenszer ismétlődik minden irányban (szemléltetésként lásd: \ref{fig:periodicboundary}. ábra). Ha egy integrálási lépés során egy részecske elhagyná a szimulációs térfogatot, az ellenkező oldalon "beléptetjük" az úgynevezett képét. Precízen arra az esetre, ha a szimulációs térfogat a $[0, ~L_x] \times [0, ~L_y] \times [0, ~L_z]$ téglatest, és a részecske $x$ koordiátája lép ki a lehetséges intervallumból:

\begin{equation}
	x \Rightarrow \begin{cases}
	\begin{array}{lr}
	x + L_x, & \rm{ha} ~x<0, \\
	x - L_x, & \rm{ha} ~x>0.
	\end{array}
	\end{cases}
	\label{eq:periodic}
\end{equation}
Valójában a periodikus határfeltétel ennél többet jelent: a részecskéknek egyszerre kell számon tartani a képeit a valódi koordinátájuk mellett, és azok is részt vehetnek párkölcsönhatásban. Ez effektíve olyan, mintha végtelen számú részecskét szimulálnánk \cite{Landau2012}. Hogy mennyi képét kell számon tartani a részecskéknek, az a párpotenciál levágási sugarától függ -- ez teszi technikailag végessé a szimulációban követett részecskék számát. Ha a levágási sugár kisebb a szimulációs doboz egy adott oldalhosszának felénél ($r_{cut} < L_i/2$), elegendő egy képet nyilvántartani abban az irányban. Ilyenkor egy atom egy másik atomnak vagy a képével, vagy a valódi atommal hat kölcsön, attól függően, melyik van hozzá közelebb. Szemléltetésként lásd a \ref{fig:periodicboundary}. ábrán a kettős végű nyilakat. 


\begin{figure}
	\centering
	\includegraphics[width=0.5\linewidth]{media/periodic_boundary}
	\caption{A periodikus határfeltétel szemléltetése \cite{Landau2012}. Középen a valódi szimulációs térfogat, ami minden irányban ismétlődik. A 4-es számú részecskékhez tartozó függőleges nyilak a határon kilépő részecske ellenkező oldalon történő beléptetését jelzik (lásd: \ref{eq:periodic}. egyenlet), a kettős végű nyilak pedig azt mutatják, hogy a párkölcsönhatások számolásakor mikor vesszük az eredeti részecskéket, illetve valamelyiknek a képét ($r_{cut} < L_i/2$ esetben).}
	\label{fig:periodicboundary}
\end{figure}


\section{Integrálási algoritmusok}

A newtoni mozgásegyenletek integrálása közönséges differenciálegyenlet-rendszer lévén egzaktul megoldható numerikusan, sokféle integrálási sémával. Molekuladinamikai szimulációk során legelterjedtebben a Verlet- és sebesség-Verlet-algoritmusokat használják, az energiamegmaradás biztosítéka és a kiemelkedő stabilitás miatt.


\subsection{Verlet-algoritmus}

Egy részecske pozíciójának diszkrét léptetése $\tau$ időlépéssel, Taylor-sorfejtést alkalmazva:

\begin{equation}
	\mathbf{r}_i (t+\tau) = \mathbf{r}_i(t) + \mathbf{v}_i(t) \tau + \frac{1}{2} \mathbf{a}_i(t) \tau^2 + \frac{1}{6} \dddot{\mathbf{r}}_i (t) \tau^3 + \mathcal{O}(\tau^4).
	\label{eq:Taylor_pos}
\end{equation}
Ugyanígy, negatív időlépést ($-\tau$) véve:

\begin{equation}
	\mathbf{r}_i (t-\tau) = \mathbf{r}_i(t) - \mathbf{v}_i(t) \tau + \frac{1}{2} \mathbf{a}_i(t) \tau^2 - \frac{1}{6} \dddot{\mathbf{r}}_i (t) \tau^3 + \mathcal{O}(\tau^4).
	\label{eq:Taylor_neg}
\end{equation}
A fenti egyenleteket összeadva ((\ref{eq:Taylor_pos})~+~(\ref{eq:Taylor_neg})):

\begin{equation}
	\mathbf{r}_i (t+\tau) + \mathbf{r}_i (t-\tau) = 2 \mathbf{r}_i(t) + \mathbf{a}_i(t) \tau^2 + \mathcal{O}(\tau^4).
\end{equation}
Ezt átrendezve, a gyorsulást átírva a következő alternatív összefüggést kapjuk a pozíció léptetésére:

\begin{equation}
	\mathbf{r}_i (t+\tau) = 2 \mathbf{r}_i(t) - \mathbf{r}_i (t-\tau) + \frac{\mathbf{F}_i(\mathbf{r}_1 (t), \mathbf{r}_2 (t) \dots \mathbf{r}_N (t))}{m_i} \tau^2 + \mathcal{O}(\tau^4).
	\label{eq:Verlet_pos}
\end{equation}
Ez az egyenlet (\ref{eq:Verlet_pos}) adja a Verlet-algoritmus alapját. Használatával a sebesség kiszámolása nélkül, negyedrendű integrálási hibával adhatjuk meg a részecskék pályáját. Az erők számolása közvetlenül a részecskék pozícióiból történik, minden időlépésben. Az algoritmus integrálási sémáját \aref{fig:verlettimeline}. ábra szemlélteti. Mivel ebben a sémában az erőnek nincs explicit időfüggése -- nem integrálással számoljuk, hanem a részecskék pozícióiból, a konzervatív potenciál kiértékelésével --, a teljes energiamegmaradás teljesül, nem befolyásolja az integrálási hiba. Noha a sebesség kiszámolása nem szükséges a pálya követéséhez, molekuladinamikai szimulációkban általában szükség van a sebesség ismeretére (a teljes mozgási energia, illetve ebből a hőmérséklet számolásához, lásd később, a 4. fejezetben). Ez a középpontidifferencia-módszerrel számolható:

\begin{equation}
	\mathbf{v}_i(t) = \frac{\dd \mathbf{r}_i(t)}{\dd t} = \frac{\mathbf{r}_i(t+\tau) - \mathbf{r}_i(t-\tau)}{2 \tau} + \mathcal{O}(\tau^2).
	\label{eq:Verlet_vel}
\end{equation}
Megjegyzendő, hogy a számolás kezdetén szükséges egy inicializáló integrálási lépés, mivel a pozíció léptetése a Verlet-algoritmussal két korábbi időlépésben felvett pozíció ismeretében történik. Az inicializáció egyszerű Euler-lépéssel:

\begin{equation}
	\mathbf{r}_i(-\tau) = \mathbf{r}_i(0) - \mathbf{v}_i(0) \tau + \frac{1}{2} \frac{ \mathbf{F}_i(\mathbf{r}_1 (0), \mathbf{r}_2 (0) \dots \mathbf{r}_N (0))}{m_i} \tau^2 + \mathcal{O}(\tau^3).
	\label{eq:Verlet_init}
\end{equation}


\begin{figure}
	\centering
	\includegraphics[width=0.5\linewidth]{media/Verlet_timeline}
	\caption{A pozíció léptetése és az erő számolása a Verlet-algoritmus során \cite{lecture}.}
	\label{fig:verlettimeline}
\end{figure}


\subsection{Sebesség-Verlet-algoritmus}

A Verlet-algoritmus másik verziója a sebesség-Verlet-algoritmus, ami explicit módon számolja a sebességet és használja fel az integráláshoz minden lépésben:

\begin{equation}
	\mathbf{r}_i (t+\tau) = \mathbf{r}_i(t) + \mathbf{v}_i(t) \tau + \frac{1}{2} \mathbf{a}_i(t) \tau^2 + \mathcal{O}(\tau^3),
	\label{eq:velVerlet_pos}
\end{equation}

\begin{equation}
	\mathbf{v}_i (t+\tau) = \mathbf{v}_i(t) + \overline{\mathbf{a}_i(t)} \tau + \mathcal{O}(\tau^2) =
	\label{eq:velVerlet_vel}
\end{equation}
\begin{equation*}
	~~~~~~~~~~~~~~~~~~~~~~~~~~~~~~ =\mathbf{v}_i(t) + \left[\frac{ \mathbf{a}_i(t+\tau) + \mathbf{a}_i(t)}{2}\right] \tau + \mathcal{O}(\tau^2).
\end{equation*}
Megjegyzendő, hogy az $\mathbf{a}_i(t)$ jelölésmód a gyorsulásra csak rövidítés: azt továbbra is az atomok pozíciójából, azaz implicit időfüggés alapján számoljuk, ahogy a (\ref{eq:Verlet_pos}) és (\ref{eq:Verlet_init}) számú összefüggésekben. A pontosságot növeli, hogy a sebesség léptetésekor a gyorsulást az időlépés elején és végén lévő konfigurációból számolt érték átlagának vesszük, illetve, hogy az így számolt sebességet használjuk a koordináta következő léptetésekor. Ez ellensúlyozza azt a hátrányt, hogy a sebesség-Verlet-algoritmus hibája a pályára alacsonyabb rendű a Verlet-algoritmushoz képest (harmadrendű vs negyedrendű), és végeredményben a két algoritmus hasonló pontosságú. Ahogy az látható, a számolás során a koordinátákat mindig előbb kell léptetnünk, mint a sebességet. A koordináták léptetéskor fontos szerepe van a periodikus határfeltételnek, hiszen módosítja az atomokra ható erőt, így a sebesség közvetlen utána következő léptetését. A határokon történő ellenőrzést tehát a pozíció léptetése után, a sebesség léptetése előtt kell elvégezni.



\section{Termodinamikai mennyiségek meghatározása}

RELAXÁCIÓ: RANDOM KEZDŐSEBESSÉG, VELOCITY SCALING

We assume that the number of particles is large enough to use statistical mechanics to relate
the results of our simulation to the thermodynamic quantities

equipartition theorem




%\begin{figure}[H]
%    \begin{center}
%    \includegraphics[width=0.5\textwidth]{media/tsneplot.png}
%    \caption{t-SNE plot for MNIST dataset \cite{tsne-article}} 
%    \label{fig:tsneplot}
%    \end{center}
%\end{figure}



\bibliographystyle{plain}
\bibliography{references}

\end{document}
