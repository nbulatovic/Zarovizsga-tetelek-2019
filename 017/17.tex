\documentclass[12pt]{article}
\setlength{\textwidth}{17cm}
\setlength{\textheight}{24cm}
\setlength{\topmargin}{-2cm}
\setlength{\footskip}{1cm}
\setlength{\evensidemargin}{0cm}
\setlength{\oddsidemargin}{0cm}
\setlength{\parindent}{0cm}

\usepackage{allrunes}
\usepackage{amsmath}
\usepackage[magyar]{babel}
\usepackage[T1]{fontenc}
\usepackage[utf8]{inputenc}
\usepackage{fixltx2e}
\usepackage{multirow}
\usepackage{cite}
\usepackage[hyphens]{url}
\usepackage[colorlinks=true,breaklinks]{hyperref}

\hypersetup{colorlinks,linkcolor=blue,urlcolor=magenta,citecolor=magenta}
%Breaks long url`s in text, while keeping it one link:

\usepackage{amsfonts}
\usepackage{amsthm}
\usepackage{amssymb}


\theoremstyle{plain}
\usepackage{graphicx}

%\usepackage{gensymb}
\usepackage{float}

% For bra-ket notation
\usepackage{braket}

%% New commands
\newcommand{\dd}{\textrm{d}}

%% Pauli matrices
\newcommand{\sigx}{\sigma_x}
\newcommand{\sigy}{\sigma_y}
\newcommand{\sigz}{\sigma_z}

\newcommand{\paulix}{
   \left( \begin{array}{cc}
      0 & 1 \\
      1 & 0
   \end{array}
   \right)
}

\newcommand{\pauliy}{
   \left( \begin{array}{cc}
      0 & -i \\
      i & 0
   \end{array}
   \right)
}

\newcommand{\pauliz}{
   \left( \begin{array}{cc}
      1 & 0 \\
      0 & -1
   \end{array}
   \right)
}


\begin{document}
\title{17. tétel: Vizualizáció}
\author{Prósz Aurél}

\maketitle

\pagenumbering{gobble}

\begin{abstract}
   \centering
   Vizualizáció - Tudományos adatok két és háromdimenziós megjelenítésének technikái, skalár-, vektor- és tenzorértékű adatok ábrázolása. Színek. Térbeli adatok megjelenítése, volometrikus és szintfelületes ábrázolás. Háromdimenziós renderelési technikák, ray casting, ray tracing. A sztereoszkópius megjelenítés technológiái. Számítógépes geometria.

   A tételkidolgozást főleg a weben elérherő anyagok alapján végeztem el, az adott rész végén megjelöltem mindenhol a forrásokat, illetve nagyobb témakör esetén, hogy hol lehet bővebben utánanézni.
\end{abstract}

\vfill

\tableofcontents

\section{Tudományos adatok két és háromdimenziós megjelenítésének technikái}
Többdimenziós tudományos adatok megjelenítésére számos módszer áll rendelkezésre, a továbbiakban ezeket szedtem össze kis magyarázat és példa keretében. 
Tekintsünk egy példa adatkészletet, például az úgynevezett \textit{Wine Quality Data Set}-et, amely letölthető innen:  \url{https://archive.ics.uci.edu/ml/index.php}. Az adatkészlet a portugál \textit{Vinho Verde} bor tulajdonságait írja le, és ezen mutatom be a vizualizációs technikákat. 


\subsection{1D és 2D vizualizáció}
Egy dimenzióban, amennyiben egy változó eloszlására vagyunk kiváncsiak, ábrázolhatjuk az adatot \textbf{hisztogramként}. Ha egyszerre több változót kívánunk ábrázolni, úgy lehetőség van őket egyenként elrendezni egy rácson. 

\begin{figure}[H]
   \centering
   \includegraphics[width=8cm, height=6cm]{media/hist.png}
   \caption{\textbf{1D adatok ábrázolása hisztogramként.}}
   \label{fig:GeneralDiagram}
\end{figure}

Egyszerű módszer az adatokon belüli mintázatok és korrelációk felfedezésére az úgynevezett korrelációs módszer kiszámítása, majd annak vizualizációja \textbf{heatmap} formájában. A heatmap esetében az adott párokhoz tartozó értékek jelzik a korreláció, vagy általános esetben bármilyen skalár értékű függvény nagyságát.

\begin{figure}[H]
   \centering
   \includegraphics[width=8cm, height=6cm]{media/kormap.png}
   \caption{\textbf{Korrelációs mátrix megjelenítése heatmap formájában.}}
   \label{fig:GeneralDiagram}
\end{figure}

Két változó közötti összefüggés megjelenítésére alkalmas a \textbf{scatterplot}, míg a hisztogramokhoz hasonlóan több változó esetén ezt is megjeleníthetjük egy rácson.

\begin{figure}[H]
   \centering
   \includegraphics[width=8cm, height=6cm]{media/pariwise.png}
   \caption{\textbf{Scatterplotok megjelenítése rácson.}}
   \label{fig:GeneralDiagram}
\end{figure}

Több dimenziós, kategorikus adat megjelenítésére szolgál az úgynevezett \textbf{multiple bar plot}, ahol egy klasszikus bar ploton tudunk megjeleníteni több változót is.

\begin{figure}[H]
   \centering
   \includegraphics[width=8cm, height=6cm]{media/multiplabar.png}
   \caption{\textbf{Multiple bar plot két változóval.}}
   \label{fig:GeneralDiagram}
\end{figure}

A \textbf{box plot} több információt tartalmazó változata a \textbf{violin plot}, amely többváltozós adatot vizualizál olyan módon, hogy az eloszlás alakja is leolvasható az outlier adatpontokkal együtt.


\begin{figure}[H]
   \centering
   \includegraphics[width=8cm, height=6cm]{media/violin.png}
   \caption{\textbf{Violin plot több változóval és kategorikus szinezéssel.}}
   \label{fig:GeneralDiagram}
\end{figure}


\subsection{3D adatvizualizáció}
Amennyiben egyszerre 3 változót szeretnénk megjeleníteni, lehetőségünk van alkalmazni a már korábban látott \textbf{scatterplot}-ot egy rácson elhelyezve, de színkódolás segítségével újabb változó értékét megjeleníthetjük.

\begin{figure}[H]
   \centering
   \includegraphics[width=8cm, height=6cm]{media/3dscatter.png}
   \caption{\textbf{2D scatterplot rácson színkóddal.}}
   \label{fig:GeneralDiagram}
\end{figure}

Ha mind a három változónk folytonos, akkor érdemes az úgynevezett \textbf{3D scatterplot}-on ábrázolni, amelyben az adatpontok mélységében tároljuk a plusz információt a kétdimenziós megjelenítéshez képest.

\begin{figure}[H]
   \centering
   \includegraphics[width=8cm, height=6cm]{media/3dcont.png}
   \caption{\textbf{3D scatterplot.}}
   \label{fig:GeneralDiagram}
\end{figure}

Magasabb dimenzióban az eddigiek analógiájára bővíthetjük a megjelenítést újabb színkódolás, forma, méret, vagy egyéb paraméterek bevezetésével. 
Bővebb információkért, illetve egzotikusabb vizualizációs technikákért érdemes megtekinteni a következő linket: \url{https://towardsdatascience.com/the-art-of-effective-visualization-of-multi-dimensional-data-6c7202990c57}

\section{Skalár-, vektor- és tenzorértékű adatok ábrázolása}

\subsection{Skalár értékű adatok ábrázolása}

Skalár értékű adatokat az előző részekben tárgyaltakon kívül ábrázolhatunk még felületeken, vagy térfogatokban olyan módon, hogy színekkel jelöljük a konkrét értékeket, ezt nevezzük \textbf{color mapping}-nak. 
Másik módszer az úgynevezett \textbf{contouring}, amely egyfajta kibővítése a color mapping-nak olyan módon, hogy a szinte azonos színű (és értékű) régiókat kontúrvonalakkal (vagy kontúrfelületekkel) elkeríti, és ezeket a területeket (vagy térfogatokat) ugyanolyan színnel színezi. Tipikus példa erre a megjelenítsére az időjárást ábrázoló térképek, ahol a konstans hőmérsékletet jellemző kontúrvonalak választják el az egyes régiókat.

\begin{figure}[H]
   \centering
   \includegraphics[width=8cm, height=6cm]{media/iso.PNG}
   \caption{\textbf{Példa a kontúrvonalakra és a kontúrfelületekre.}}
   \label{fig:GeneralDiagram}
\end{figure}

\subsection{Vektor értékű adatok ábrázolása}
Vektor értékű adatok adatok ábrázolását a legegyszerűbb módon úgy valósíthatjuk meg, hogy minden egyes adatponthoz társítunk egy vonalat, amelynek meghatározott iránya és nagysága van. A technika neve \textbf{hedgehog}, a keletkező vizualizáció kinézete miatt. A megjelenítést tovább lehet finomítani a vonalak színezésével és méretük változtatásával. Ezen kívül az egyszerű vonalak helyett alkalmazhatunk 2D és 3D irányított objektumokat is, amelyek többdimenziós adatok megjelenítésénél lehet hasznos, mert könnyebb a jelölt irányt megállapítani az alkalmazásukkal.

\begin{figure}[H]
   \centering
   \includegraphics[width=8cm, height=6cm]{media/glyph.PNG}
   \caption{\textbf{Példa az egyszerű vonalas vektorábrázolásra, illetve a bonyolultabb 2D és 3D objektumok használatára.}}
   \label{fig:GeneralDiagram}
\end{figure}

A vektor értékű adatok ábrázolásánál figyelni kell, hogy 3 dimneziós ábrázolásoknál sokszor nehéz pontosan megállapítani a vektorok irányítottságát, hiszen a 3D térből vetítjük le a megjelenítést a 2D térbe. Ezen kívül fontos, hogy ne jelenítsünk meg egyszerre túl sok adatpontot, mert a vektorok összemosódhatnak és értelmezhetetlenné válik az ábra. 

\subsection{Tenzor értékű adatok ábrázolása}
A tenzor értékű adatok megjelenítése komplex feladat és jelenleg is aktívan kutatott terület. Használatuk kiemelkedő biológiai adatok (molekuláris diffúzió mérése) és mérnöki területek (feszültég tenzor) esetében. Ahhoz, hogy vizualizáljuk a tenzor értékű adatokat, minden egyes adatpont helyén 3 ortogonális vektor vizualizációjára van szükségünk. Ehhez felhasználhatjuk a vektor értékű adatok ábrázolásánál használt módszereket. 
A feszültég tenzor esetében például a vizualizálni kívánt tenzorokat reprezentálhatjuk 3X3 mátrixok segítségével. Ebből számolható 3 sajátérték és 3 ortogonális sajátvektor. A sajátvektorok hossza tartalmazhatja az információt az adott irányba keletkező feszültség értékéről, míg további színkódolással a mennyiség előjelét is meg lehet jeleníteni. Ez a módszer csak abban az esetben használható eredményesen, ha az adatpontok nincsenek túl közel egymáshoz és a feszültség homogén módon oszlik el az anyagban. 



\begin{figure}[H]
   \centering
   \includegraphics[width=8cm, height=6cm]{media/stress.png}
   \caption{\textbf{Példa a feszültség tenzor vizualizációjára.}}
   \label{fig:GeneralDiagram}
\end{figure}

További vizualizációs módszer az úgynevezett \textbf{colormap} használata, mikor egy ábrán csak a legnagyobb sajátértékhez tartozó sajátvektorokat jelenítjük meg, irányukat pedig színekkel kódoljuk. 

\begin{figure}[H]
   \centering
   \includegraphics[width=8cm, height=6cm]{media/colormap.PNG}
   \caption{\textbf{Tenzor vizualizáció colormap módszerrel.}}
   \label{fig:GeneralDiagram}
\end{figure}

Végül egy egyszerű technika, amelyel a mátrix összes komponensét tudjuk egyszerre vizualizálni: elkészítünk egy 3X3-as mozaik képet, amelynek minden egyes eleme az adott vizualizáció, olyan módon, hogy csak az adott mátrixelem értékeit jelenítjük meg benne.

\begin{figure}[H]
   \centering
   \includegraphics[width=8cm, height=6cm]{media/brain.PNG}
   \caption{\textbf{Mátrixelemek vizualizációja. Az egyes képek az adott elemhez tartozó intenzitásértékeket kódolják.}}
   \label{fig:GeneralDiagram}
\end{figure}

Bővebben:
\url{http://www.inf.ed.ac.uk/teaching/courses/vis/lecture\_notes/lecture14.pdf} és  \url{https://booksite.elsevier.com/samplechapters/9780123875822/9780123875822.PDF}


\section{Színek}
Tudományos publikációkban rendkívül fontos szerepe van a megfelelő színválasztásnak. Ehhez előre definiált színpaletták állnak rendelkezésre, a következőkben ezeket tekintem át. Általános alapelv a színválasztásnál, hogy a megjelenített adatokat egyértelművé tegye, ne torzítsa el, és az esetleges színtévesztő olvasók számára is problémamentes legyen a megjelenítés.

\subsection{Szekvenciális adatok színezése}

Amennyiben olyan adatot kívánunk megjeleníteni, amely értékei folytonosak és sorba rendezhetőek nagyság szerint, úgy érdemes az úgynevezett szekvenciális paletták közül válogatni. Ennek előnye, hogy a legtöbb ember a sötétebb színeket, például a feketét a "több", vagy "sűrűbb" adathoz társítja, míg a világosabb színek az ellenkező hatást érik el. 

\begin{figure}[H]
   \centering
   \includegraphics[width=8cm, height=6cm]{media/sequential.png}
   \caption{\textbf{Szekvenciális paletták}}
   \label{fig:GeneralDiagram}
\end{figure}

\begin{figure}[H]
   \centering
   \includegraphics[width=8cm, height=6cm]{media/map.png}
   \caption{\textbf{Egy példa a szekvenciális adatok színezésére.}}
   \label{fig:GeneralDiagram}
\end{figure}
\subsection{Divergens adatok színezése}
Ha az adataink divergens struktúrájúak, tehát tartozik hozzá egy nullpont, amihez képest lehet az adatok értéke nagyobb, vagy kisebb, akkor célszerű a divergens színpalettákból válogatni. Ezek olyan paletták, amelyek két vége tendál a sötétebb színek felé, így mutatva, hogy azok a részei a "jobban" negatívak, vagy pozitívak.

\begin{figure}[H]
   \centering
   \includegraphics[width=8cm, height=6cm]{media/diverging.png}
   \caption{\textbf{Divergens paletták}}
   \label{fig:GeneralDiagram}
\end{figure}

\subsection{Kategorikus adatok színezése}
Kategorikus adatok esetében, amelyek nem állíthatóak sorba, törekednünk kell arra, hogy egymástól jól megkülönböztethető színeket használjunk, illetve, hogy egy ábrázoláson csak megfelelő mennyiségű kategóriát szerepeltessünk, hogy ne legyenek összekeverhetőek az adatpontok.



\begin{figure}[H]
   \centering
   \includegraphics[width=8cm, height=6cm]{media/categorical.png}
   \caption{\textbf{Kategorikus paletták}}
   \label{fig:GeneralDiagram}
\end{figure}

\subsection{Egyéb szempontok}

Fontos szempont, hogy az adatok megjelenítésénél gondoljunk  színtévesztő olvasókra. Ehhez általános szabály, hogy pirosat és zöldet soha ne használjunk együtt, mert a színtévesztés leggyakoribb fajtája (8\% férfiaknál, 0.5\% nőknél) pont ezt a két színt teszi nehezen megkülönböztethetővé. Ezen szabály betartása mellett rendelkezésre állnak kifejezetten színtévesztőknek kifejlesztett színpaletták is, ezek használata is indokolt lehet. 

\begin{figure}[H]
   \centering
   \includegraphics[width=8cm, height=6cm]{media/cwheel-polar.png}
   \caption{\textbf{Színtévesztés szimulálása.}}
   \label{fig:GeneralDiagram}
\end{figure}

\begin{figure}[H]
   \centering
   \includegraphics[width=8cm, height=6cm]{media/colorb.png}
   \caption{\textbf{Színtévesztők számára is könnyen értelmezhető színpaletták}}
   \label{fig:GeneralDiagram}
\end{figure}



\section{Térbeli adatok megjelenítése, volometrikus és szintfelületes ábrázolás.}
Térbeli adatokat vizualizálhatunk olyan módon, hogy veszünk egy skalármezőt (3D tér pontjaihoz skalár értékeket rendelünk), és valamilyen fényszóró anyag analógiáját használva rendereljük a rajta keresztül haladó fénysugarakat. Ilyen módon az anyag optikai, általunk előre definiált tulajdonságai függvényében láthatunk az objektum mélyére. Részletes levezetést itt nem mellékelek, de elérhető az alábbi prezentációban: \url{http://cg.iit.bme.hu/portal/oktatott-targyak/szamitogepes-grafika-es-kepfeldolgozas/terfogat-vizualizacio}, illetve sematikusan a lenti ábrán.

\begin{figure}[H]
   \centering
   \includegraphics[width=12cm, height=8cm]{media/terf1.PNG}
   \caption{\textbf{Térfogatvizualizáció skalármezővel.}}
   \label{fig:GeneralDiagram}
\end{figure}

\begin{figure}[H]
   \centering
   \includegraphics[width=12cm, height=8cm]{media/terf2.PNG}
   \caption{\textbf{Térfogati modell.}}
   \label{fig:GeneralDiagram}
\end{figure}

\subsection{Szintvonalas ábrázolás alapfogalmai}
Az alapfogalmak bevezetéséhez a térképészetet, mint analógiát használja fel a hivatkozott írás. 
A szintvonal egy választott alapszintfelülettől, a középtengerszinttől azonos magasságban lévő tereppontokat összekötő görbe vonal. Úgy is fogalmazhatjuk, hogy a nehézségi erőtérnek az alapszintfelülettől egy adott magasságban lévő szintfelületének és a terepnek a metszetvonala.

Szintvonalas magasságábrázolást topográfiai térképeken csak azokon a természetes terepalakulattal rendelkező tereprészeken alkalmazunk, ahol egy vízszintes értelmű ponthoz csak egy magasságérték tartozik. Ez a feltétel a gyakorlatban még tovább szigorodik azzal, hogy pl. a terep járhatósága, a szintvonalak kirajzolhatósága miatt, csak a 40$^{\circ}$-nál kisebb lejtőszögű területeken alkalmazunk szintvonalakat. Ezeket a feltételeket figyelembe véve, a szintvonalakról elmondhatjuk, hogy:

\begin{itemize}
   \item önmagukba visszatérő görbék
   \item egymást sohasem keresztezik
   \item minél meredekebb a terep, annál közelebb haladnak egymás mellett
   
\end{itemize}


Ilyen vonal a terepen végtelen sok elképzelhető, a térképeken mindet nem tudnánk ábrázolni, és nem is fejezne ki semmit. Ezek közül a térképeken csak azokat tüntetjük fel és nevezzük őket alapszintvonalaknak, melyek az alapszintfelülettől kezdve, egymástól egy előre megadott távolságra, az ún. alapszintköz távolságra helyezkednek el egymástól. Az alapszintköz megválasztása függ a térkép méretarányától és a az ábrázolt terepen található domborzati formák jellegétől, meredekségétől.

A szintvonalas magasságábrázolás előnyei:

\begin{itemize}
   \item az abszolút, tengerszint feletti magassága minden terepi pontnak meghatározható
   \item a lejtősség mértékét és irányát jól ábrázolja
   \item a lejtősség mértéke és iránya számszerűen is meghatározható
   \item a domborzati idomok formáját jól kifejezi
\end{itemize}

A szintvonalas magasságábrázolás hátrányai:

\begin{itemize}
   \item nem elég plasztikus, a kiemelkedő és a bemélyedő idomok szemlélet alapján nem választhatók szét
   \item csak azok az idomok fejezhetők ki, melyeket szintvonal metsz, a szintvonalak közötti magasságváltozások nem ábrázolhatók

\end{itemize}


\url{https://regi.tankonyvtar.hu/hu/tartalom/tamop425/0027\_TOP4/ch01s03.html}

\section{Háromdimenziós renderelési technikák, ray casting, ray tracing.}
Kezdésnek tekintsük át, hogy hogyan lehet leképezni 3D objektumokat egy 2D felületre. Az alább lévő ábrán látható, hogy ehhez szükségünk van az objektumra, egy felületre, majd egy pontra, amelyet úgy szerkesztünk, hogy a kapott piramis teteje lesz a pont, a talapzata pedig a felület. A csúcstól a felület felé húzott egyenessel párhuzamos a látósugarunk, a felületet pedig a képsík lesz. Erre a képsíkra fogjuk vetíteni a 3D objektum képét olyan módon, hogy a piramis csúcspontjából húzunk vonalakat az objektum különböző részeihez, és a vonalak és a sík metszéspontjai fogják adni a vetített képet.

\begin{figure}[H]
   \centering
   \includegraphics[width=8cm, height=6cm]{media/vantagepoint.png}
   \caption{\textbf{A számítógépes képalkotás alapja.}}
   \label{fig:GeneralDiagram}
\end{figure}

A képelkotáshoz szükség van még egy fényforrásra is, amely megvilágítja az objektumokat. Ezután több különböző módszer áll rendelkezésre annak érdekében, hogy valósághű képet hozhassunk létre. 

\subsection{Raszterelés}

A rasztergrafika, másként pixelgrafika olyan digitális kép, ábra, melyen minden egyes képpontot (pixelt) önállóan definiálunk. 

Előnyei: Egyszerű adatszerkezet; egyszerű algoritmus; gyors feldolgozás; fototechnikai trükköknél jól alkalmazható.

Hátrányai: az adatállomány nagy méretű; rögzített felbontás; nagyításnál a minőség romlik.

A pixelgrafikus rajzolóprogramok a képeket mátrix-szerűen elrendezett képpontokból, pixelekből építik fel. A sorokat és oszlopokat alkotó képpontok különböző színűek lehetnek, ezekből a pontokból áll össze a rajz. A bitmap grafika (vagy rasztergrafika) egy kép tartalmát egy négyzetrácson elhelyezkedő színes képpontok összességeként, ún. pixelekkel írja le. Ahogy a képen látható, a falevél képét képpontjai helyének és a képpontok színértékeinek tárolásával hozzuk létre úgy, mintha egy mozaik kockáit raknánk egymás mellé. A pixelekből álló képet a kép felépítésére utalva bittérképnek is nevezik. A bittérképek egyik legfontosabb tulajdonsága a felbontás. A kép minőségét több felbontás-típus egyszerre határozza meg.

Ezzel szemben a vektorgrafikus rajzolóprogramok a képek felépítésére egyszerű alakzatokat (téglalap, ellipszis, sokszög, stb.) és ún. Bézier görbéket (csomópontokkal, a csomópontok közt húzott görbékkel és érintőszakaszokkal felépített görbéket) használnak. A vektoros képkészítésnek számos előnye van, de vannak korlátai is. Mivel a képek nem képpontokból állnak, tetszőlegesen nagyíthatók és kicsinyíthetőek, a végeredmény minősége csak a képmegjelenítő eszköztől függ. Lényeges szempont, hogy mennyi hely szükséges a program által előállított állományok tárolására.

Egy bittérképnél - egy pixelgrafikus rajzolóprogrammal készített grafikánál - természetes, hogy a kép méretével a képfájl mérete is növekszik, hiszen több képpont adatait kell tárolni. Mivel a vektorgrafikus rajzolóprogramok a képeket csomópontok segítségével építik fel, a képfájlok méretét a csomópontok és görbék száma határozza meg: minél több csomópont szükséges a kép leírásához - tehát minél több görbéből áll a kép -, annál nagyobb a vektoros állomány mérete. Mivel a kép nagyításával, illetve kicsinyítésével nem változik a csomópontok száma, természetes, hogy nem változik az állomány mérete sem. Bonyolultabb grafikák esetében (pl. tervezőprogramok, 3D modellező programok) több MB méretű vektoros állomány is előállítható.

Egy vektorokból álló objektumokkal felépített képen minden objektum kitölthető valamilyen színnel, viszont - mivel az objektumok jól elkülöníthető görbékből állnak - nincs lehetőségünk fototechnikai eljárások (elmosás, élesítés, homályosítás) használatára. Ezek a műveletek csak pixelgrafikus rajzolóprogramokkal végezhetők el. A mai rajzolóprogramok természetesen képesek a vektorgrafikus rajzokat pixeles formátumra konvertálni, amelyen azután további módosításokat végezhetünk. 

Raszterelés során egy vektorgrafikát alakítunk rasztergrafikává. A legegyszerűbb esetben 3D objektumokat poligonokkal reprezentálhatunk, ezért a leggyakoribb raszterelési probléma a háromszög raszterelése. 



\begin{figure}[H]
   \centering
   \includegraphics[width=8cm, height=6cm]{media/rasterelott.PNG}
   \caption{\textbf{Raszterelés előtt}}
   \label{fig:GeneralDiagram}
\end{figure}

\begin{figure}[H]
   \centering
   \includegraphics[width=8cm, height=6cm]{media/rasterutan.PNG}
   \caption{\textbf{Raszterelés után}}
   \label{fig:GeneralDiagram}
\end{figure}


\subsection{Scanline rendering}

Scanline renderelés során a látható felületeket sorról sorra határozzuk meg, ellentétben más módszerekkel (amik lehetnek pixelről pixelre és poligonnról poligonra). 
Előnye, hogy kevesebb összehasonlítást kell elvégeznünk, illetve kevesebb élt kell eltárolnunk egyszerre a memóriában, hiszen csak az adott scanline-hoz tartozó poligonokkal kell foglalkoznunk.


bővebben: \url{https://www.cs.drexel.edu/~david/Classes/CS5%36/Lectures/L-16_ScanlineRendering.pdf}

\subsection{Ray tracing - Ray casting}

Tudjuk, hogy ahhoz, hogy a képalkotás létrejöhessen, a fényforrásból elsőnek az objektumra kell jutniuk a fotonoknak, majd onnan verődve az érzékelőre. Ehhez végigszámolthatjuk az összes foton útját a fényforrástól a szenzorig (ez az úgynevezett \textbf{forward tracing} technika), de mivel a fotonok többsége nem az érzékelőre fog veródni az objektumról, ezért ez nagyon számításigényes és nem hatékony eljárás. Ezzel szemben, ha megfordítjuk a kiértékelés menetét, olyan módon, hogy fotonokat bocsátunk ki az érzékelőből, és visszaszámoljuk, hogoy milyen úton jut el a fényforrásig. Ennek részletesebb menete úgy néz ki, hogy kibocsátunk egy fotont a képsík felé, majd megvizsgáljuk, hogy a foton útjával párhuzamosan húzott vonal (\textbf{primary ray}) ér-e objektumot. Ha igen, akkor az érintési ponttól szerkesztünk egy újabb egyenest a fényforrás felé, és vizsgáljuk, hogy ennek a szerkesztett egyenesnek van-e az útjában újabb objektum. Ezt a fotont, vagy sugarat nevezzük \textbf{shadow ray}-nek és arra használjuk, hogy megállapítsuk, hogy az adott objektum adott része árnyékban van-e. Az elv részletesebben megtekinthető az alsó ábrán.

\begin{figure}[H]
   \centering
   \includegraphics[width=8cm, height=6cm]{media/shadow1.PNG}
   \caption{\textbf{Képalkotás, ha az objektum adott része nincs árnyékban.}}
   \label{fig:GeneralDiagram}
\end{figure}

\begin{figure}[H]
   \centering
   \includegraphics[width=8cm, height=6cm]{media/shadow2.PNG}
   \caption{\textbf{Képalkotás, ha az objektum adott része árnyékban van.}}
   \label{fig:GeneralDiagram}
\end{figure}

\begin{figure}[H]
   \centering
   \includegraphics[width=8cm, height=6cm]{media/Ray_trace_diagram.png}
   \caption{\textbf{Ray tracing összefoglalva.}}
   \label{fig:GeneralDiagram}
\end{figure}

\url{https://www.scratchapixel.com/lessons/3d-basic-renderi%ng/ray-tracing-overview}

A ray-casting a ray-tracing egy egyszerűbb, de sokkal gyorsabb formája, ahol csak a közvetlen fénysugarakat vesszük figyelembe.

\begin{figure}[H]
   \centering
   \includegraphics[width=8cm, height=6cm]{media/ray-casting-diagram.jpg}
   \caption{\textbf{Ray-casting.}}
   \label{fig:GeneralDiagram}
\end{figure}

\section{A sztereoszkópius megjelenítés technológiái}

Sztereoszkópiának hívunk bármilyen olyan technológiát, amely lehetőséget ad egy 2D kép olyan módon való megjelenítésére, amely valódi 3D képnek tűnő hatást kelt. Fotók, vagy videók esetében ezt úgy érik el, hogy különböző képet mutatnak a különböző szemeknek. 
A térlátás abból adódik, hogy az emberi szemek közelítőleg 5 cm-re helyezkednek el egymáshoz képest, így egy adott jelenséget a két szem kicsit más szögből látja. Lehetőség van ennek analógiájára olyan kamerát készíteni, amely két lencsével rendelkezik, reprodukálva az emberi látás sajátosságait. Alapvetően kétfajta technikát különböztetünk meg: a sztereoszkópikus és az autosztereoszkópikus megjelenítéseket. Az elsőnél a megfigyelőnek szüksége van valamilyen eszköz, például speciális szemüveg viselésére a 3D hatás megtapasztalására, míg a másodiknál nincs. 
\subsection{Anaglif megjelenítés}
A sztereoszkópikus technológia felbontható több csoportra. Az első ilyen módszer az úgynevezett anaglif megjelenítés. Ennek során egy képet két különböző szögből vesznek fel, majd két különböző színnel látnak el, amelyet hasonló színezésű szemüveggel (egy szín az egyik, másik szín a másik szemnek) vizsgálva előhozható a 3D megjelenítés. Általában a két használt szín a piros és a kék. Előnyei a technikának, hogy nagyon egyszerűen lehet előállítani a kompozit képet, és olcsó a hozzá tartozó szemüveg. Ezen kívül fizikailag is kinyomtatható a kép, a hatás úgy is megmarad. Hátrányai, hogy nehéz valódi színeket megjeleníteni rajta, a keletkező kép gyakran homályos.

\begin{figure}[H]
   \centering
   \includegraphics[width=8cm, height=6cm]{media/stereo.PNG}
   \caption{\textbf{Anaglif kép.}}
   \label{fig:GeneralDiagram}
\end{figure}



\subsection{Polarizációs szemüvegek}

Ez a technika hasonló az anaglif megjelenítéshez, de itt színkódolás helyett különböző polarizációs szűrőkön vezetik át a két szemhez tartozó képet. Előnye az előzőhöz képest a jobb színhűség, de hátránya, hogy a megjelenő kép homályos és fényszegény a szűrők fényelnyelése miatt.

\subsection{Shutter szemüvegek}

Ennek a technika során a speciális szemüveg periodikusan kitakarja egyszer az egyik, másszor a másik szem képét, így a kijelző videokártyájával megfelelően szinkronizálva, majd azon megfelelő időben megjelenítve a képeket elérhető a 3D hatás. Előnye, hogy sokkal meggyőzőbb képet ad az előző két technikával szemben, de hátránya, hogy a speciális szemüvegek gyakran nagyon drágák, és értelemszerűen a keletkező képet kinyomtatni se lehet. 
\subsection{Autosztereoszkópikus technikák}
Az autosztereoszkópikus technikák is felbonthatóak több csoportra. Alapvetően mindegyik technikára igaz, hogy akkor válik láthatóvá a 3D kép, ha a néző szemei úgynevezett \textit{látóablakokban} helyezkednek el, amelyek a kijelzőtől bizonyos szögekben elhelyezkedő területek. Attól függően, hogy ezek a területek hogy helyezkednek el, megkülönböztethetünk 3 módszert.
\textbf{ Fix látóablak}: A látóablak egy fix szögben helyezkedik el a képernyő előtt, így csak ebből a jól meghatározott pozícióból látszódik a 3D kép. Praktikusan ez azt jelenti, hogy egyszerre csak egy néző használhatja.   

\textbf{Néző követés}: Egy kamera segítségével a néző arcát lehetséges követni, így szinte bármilyen irányból megjeleníthető a 3D kép, de továbbra is csak egy néző számára.  
\textbf{Többszörös látóablak}: A képernyő több látóablakot hoz létre fix helyeken, így több ember is használhatja, illetve viszonylag egyszerű átpozícionálnia magát a megfigyelőnek más nézőpontba.  

\begin{figure}[H]
   \centering
   \includegraphics[width=8cm, height=6cm]{media/stereo_m.PNG}
   \caption{\textbf{Többszörös látóablakos technika.}}
   \label{fig:GeneralDiagram}
\end{figure}

A konkrét technológiai megvalósítása a fenti módszereknek legtöbbször LED kijelzőkkel történik, amelyekre plusszban a 3D megvalósításhoz szükséges komponenseket adják hozzá. Ennek legegyszerűbb példája az úgynevezett \textbf{Parralax gát technika}, amely során átlátszatlan, vertikális szálak sorozatával blokkolják a megfelelő pixelekből érkező fényt. Hátránya a módszernak, hogy ezek a szálak csökkentik a fényerőt. 

\begin{figure}[H]
   \centering
   \includegraphics[width=8cm, height=6cm]{media/paralax.PNG}
   \caption{\textbf{Parralax gát technika.}}
   \label{fig:GeneralDiagram}
\end{figure}


\url{https://www.cg.tuwien.ac.at/courses/Seminar/WS2006/stereo_techniques.pdf}

\section{Számítógépes geometria}
A számítógépes geometria témája bőven nagyobb terjedelmű, mint amennyit ebben a kidolgozásban be lehet mutatni, így hivatkozásnak adok meg lejjebb hasznos forrásokat.  

Alapvetően a számítógépes geometria a következő objektumok digitális reprezentálásával, velük kapcsolatos algoritmusokkal foglalkozik:

\begin{itemize}
   \item pontok
   \item pontfelhők
   \item háromszögek
   \item háromszöghálók
   \item görbék
   \item felületek
   \item testek
   
\end{itemize}

Az ezekkel kapcsolatos műveletek lehetnek például: pontfelhők egyesítése, szűrése, egyszerűsítése, háromszögelés (háló generálás), görbe interpoláció és approximáció, felület interpoláció és approximáció és celluláris (voxel) reprezentáció.

Bővebben:
\url{http://cg.iit.bme.hu/portal/sites/default/files/education/computer%20graphics%20%28bme%29/incremental%203d%20rendering/abmeincr.pdf?fbclid=IwAR0s_uDL9wo6ym7BNJ7gIyAf-0WeTthBdS8sKqI3tUel4AC2JWcQN72zfPY}
\url{http://cg.iit.bme.hu/portal/en/Education/Computer%20Graphics%20%28BME%29/Increm%ental%203D%20rendering?fbclid=IwAR0s_uDL9wo6ym7BNJ7gIyAf-0WeTthBdS8sKqI3tUel4AC%2JWcQN72zfPY}
\url{https://slideplayer.hu/slide/2146461/} 
\url{https://docplayer.hu/105823649-3d-szamitogepes-geometria-es-alakzatrekonstrukcio.html}

\end{document}
