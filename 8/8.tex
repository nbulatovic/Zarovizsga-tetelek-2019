\documentclass[12pt]{article}
\setlength{\textwidth}{17cm}
\setlength{\textheight}{24cm}
\setlength{\topmargin}{-2cm}
\setlength{\footskip}{1cm}
\setlength{\evensidemargin}{0cm}
\setlength{\oddsidemargin}{0cm}
\setlength{\parindent}{0cm}

\usepackage{allrunes}
\usepackage{amsmath}
\usepackage[english]{babel}
\usepackage[T1]{fontenc}
\usepackage[utf8]{inputenc}
\usepackage{fixltx2e}
\usepackage{multirow}

\usepackage[hyphens]{url}
\usepackage[unicode,colorlinks=true,breaklinks]{hyperref}
%\usepackage[dvips]{hyperref}
%should display links, but it does not work with \H accent
%and formulas in section titles

\hypersetup{colorlinks,linkcolor=blue,urlcolor=magenta,citecolor=magenta}
%Breaks long url`s in text, while keeping it one link:

\usepackage{amsfonts}
\usepackage{amsthm}
\usepackage{amssymb}


\theoremstyle{plain}
\usepackage{graphicx}

%\usepackage{gensymb}
\usepackage{float}

% For bra-ket notation
\usepackage{braket}

%% New commands
\newcommand{\dd}{\textrm{d}}

%% Pauli matrices
\newcommand{\sigx}{\sigma_x}
\newcommand{\sigy}{\sigma_y}
\newcommand{\sigz}{\sigma_z}

\newcommand{\paulix}{
    \left( \begin{array}{cc}
        0 & 1 \\
        1 & 0
    \end{array}
    \right)
}

\newcommand{\pauliy}{
    \left( \begin{array}{cc}
        0 & -i \\
        i & 0
    \end{array}
    \right)
}

\newcommand{\pauliz}{
    \left( \begin{array}{cc}
        1 & 0 \\
        0 & -1
    \end{array}
    \right)
}


\begin{document}
\title{8th exam item}
\author{András Mátyás Biricz}

\maketitle


\newpage
\begin{abstract}
    Jelfeldolgozás és idősor-analízis – Fourier-módszerek, FFT, a spektrum és a spektrogram, az átviteli és ablakfüggvények, Wiener-szűrő. Korrelációs függvények, a Wiener–Hin-csin-tétel és a teljesítményspektrum. Konvolúció és dekonvolúció. Szűrők analóg és digitális megvalósítása, RLC-körök, FIR- és IIR-szűrők.

	Signal processing and analysis of time series - Fourier methods, FFT, spectrum and spectrogram, transfer- and window functions, Wiener-filter. Correlation functions, Wiener-Khinchin theorem, power spectrum. Convolution, deconvolution. Realization of analog and digital filters, RLC-circuit, FIR and IIR filters. 

\end{abstract}

\section{Introduction}

time series analyses and intro


\section{Fourier methods}

basics

\subsection{Fast Fourier Transform}


\subsection{Spectrum, spectrogram}


\section{Transfer and window functions}



\section{Correlation functions}



\section{Convolution}



\section{Filters}


\subsection{Analog}


\subsection{Digital}


\subsection{RLC circuit}


\subsection{FIR}


\subsection{IIR}


\subsection{Wiener-filter}



\section{Conclusion}

\begin{thebibliography}{Nature}
%\bibliography{references}

\bibitem{Deeplea_Goodfellow}\hypertarget{Deeplea_Goodfellow}{}
Ian Goodfellow and Yoshua Bengio and Aaron Courville (2016). \textit{Deep learning}. MIT Press. \url{http://www.deeplearningbook.org}

\end{thebibliography}




\end{document}
